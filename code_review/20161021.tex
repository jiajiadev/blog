\section{2016-10-21}

\subsection{命名规则}

\begin{itemize}
  \item 变量命名使用名词
  \item 函数命名使用动词,或者动词+名词的格式
  \item 变量命和函数命全部用小写字母,如果包含多个单词,单词之间使用下划线“{\_}”分割 
  \item 变量名和函数名不使用拼音,例如,index.js中使用reli,替换为heat{\_}map。
  \item 变量名和函数名要有实际意义,例如, index.js中使用gender
\end{itemize}

\subsection{编码}

\begin{itemize}
  \item python文件编码:.py文件的头部是否标明编码信息, 例如 {\#} -*- coding: utf-8 -*-
  \item html文件编码:.html文件是否标明编码格式,例如 <meta charset="utf-8" />
  \item javascript文件编码:
\end{itemize}

\subsection{缩进}

\begin{itemize}
  \item 统一代码的缩紧格式(包括python、html、javascript、css的代码):每次缩进2个空格,注意不要使用Tab
\end{itemize}

\subsection{去除冗余}

\begin{itemize}
  \item python文件:import是否有冗余
  \item python文件:是否有冗余的函数,即定义后没有被调用的函数
  \item html文件:引入的js,css文件是否有冗余
  \item 冗余函数:例如函数内部只有一行,且该行调用了其它函数
\end{itemize}

\subsection{语法的检查}

\begin{itemize}
  \item html文件:标签是否匹配,例如<div>与</div>的个数是否匹配
  \item html文件:网页布局是否正确,例如分几行,分几列,每部分的宽度、高度等等
  \item html文件:检查js库、css库,是否有漏掉的
  \item 函数式编程:对于数据集的操作,尽量不要使用for等循环结构,使用map()、filter()、reduce()等函数式编程方法
\end{itemize}


\subsection{文档}

\begin{itemize}
  \item 函数:解决的问题、流程图
  \item 网页:网页的布局,例如,网页分为几个容器,每个容器的位置、宽度、高度等
  \item 使用手册:告诉用户如何使用
\end{itemize}

\subsection{团队合作}

\begin{itemize}
  \item 每天下午5点到6点,交换检查别人的代码
  \item 下班之前将自己的代码、文档提交到Repo
\end{itemize}
