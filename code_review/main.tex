\documentclass[a4paper,  11pt]{article}

%--------------------------
% 中文字体的支持
%--------------------------
\usepackage{xeCJK}
\setCJKmainfont[BoldFont={SimHei}, ItalicFont={STKaiti}]{STSong}
\setCJKsansfont[BoldFont={STHeiti}]{STXihei}
\setCJKmonofont{STFangsong}

%--------------------------
% 数学符号公式
%--------------------------
\usepackage{latexsym}
\usepackage{amsmath} % AMS Latex宏包
\usepackage{amssymb}
\usepackage{amsbsy}
\usepackage{amsthm}
\usepackage{amsfonts}
\usepackage{mathrsfs} % 英文花体字体
\usepackage{bm} % 数学公式中的黑斜体
\usepackage{relsize} % 调整公式字体大小:\mathsmaller
\usepackage{caption2} % 浮动图形和表格标题样式

%--------------------------
% 图形
%--------------------------
\usepackage{epsfig}
%\usepackage{CJKutf8}
\usepackage{graphicx}
\usepackage[unicode]{hyperref}
\usepackage{xcolor}
\usepackage{cite}
\usepackage{indentfirst}

%--------------------------
% 设置字号
%--------------------------
\newcommand{\chuhao}{\fontsize{42pt}{\baselineskip}\selectfont}
\newcommand{\xiaochuhao}{\fontsize{36pt}{\baselineskip}\selectfont}
\newcommand{\yihao}{\fontsize{28pt}{\baselineskip}\selectfont}
\newcommand{\erhao}{\fontsize{21pt}{\baselineskip}\selectfont}
\newcommand{\xiaoerhao}{\fontsize{18pt}{\baselineskip}\selectfont}
\newcommand{\sanhao}{\fontsize{15.75pt}{\baselineskip}\selectfont}
\newcommand{\sihao}{\fontsize{14pt}{\baselineskip}\selectfont}
\newcommand{\xiaosihao}{\fontsize{12pt}{\baselineskip}\selectfont}
\newcommand{\wuhao}{\fontsize{10.5pt}{\baselineskip}\selectfont}
\newcommand{\xiaowuhao}{\fontsize{9pt}{\baselineskip}\selectfont}
\newcommand{\liuhao}{\fontsize{7.875pt}{\baselineskip}\selectfont}
\newcommand{\qihao}{\fontsize{5.25pt}{\baselineskip}\selectfont}

%--------------------------
% 设置section属性
%--------------------------
\usepackage{titlesec}
\setcounter{secnumdepth}{4}

\makeatletter
\renewcommand\section{\@startsection{section}{1}{\z@}%
	{-1.5ex \@plus -.5ex \@minus -.2ex}%
	{.5ex \@plus .1ex}%
{\normalfont\sihao\CJKfamily{hei}}}
\makeatother

%--------------------------
% 设置subsection属性
%--------------------------
\makeatletter
\renewcommand\subsection{\@startsection{subsection}{1}{\z@}%
	{-1.25ex \@plus -.5ex \@minus -.2ex}%
	{.4ex \@plus .1ex}%
{\normalfont\xiaosihao\CJKfamily{hei}}}
\makeatother

%--------------------------
% 设置subsubsection属性
%--------------------------
\makeatletter
\renewcommand\subsubsection{\@startsection{subsubsection}{1}{\z@}%
	{-1ex \@plus -.5ex \@minus -.2ex}%
	{.3ex \@plus .1ex}%
{\normalfont\xiaosihao\CJKfamily{hei}}}
\makeatother


%--------------------------
% 设置行首缩进两个字
%--------------------------
\makeatletter
\let\@afterindentfalse\@afterindenttrue
\@afterindenttrue
\makeatother
\setlength{\parindent}{2em}  %中文缩进两个汉字位


%--------------------------
% 设置页面边距
%--------------------------
\addtolength{\topmargin}{-54pt}
\setlength{\oddsidemargin}{0.63cm}  % 3.17cm - 1 inch
\setlength{\evensidemargin}{\oddsidemargin}
\setlength{\textwidth}{14.66cm}
\setlength{\textheight}{24.00cm}    % 24.62

%--------------------------
% 设置行间距和段落间距
%--------------------------
\linespread{1.4}
% \setlength{\parskip}{1ex}
\setlength{\parskip}{0.5\baselineskip}

%--------------------------
%  Math
%--------------------------
%\usepackage{latexsym}
%\usepackage{amsmath} 
%\usepackage{amssymb}
%\usepackage{amsbsy}
%\usepackage{amsthm}
%\usepackage{amsfonts}
%\usepackage{mathrsfs}
%\usepackage{bm}
%\usepackage{relsize}
%\usepackage{caption2}

%--------------------------
%  graph
%--------------------------
\usepackage{epsfig}
\usepackage{graphicx}
\usepackage[unicode]{hyperref}
%% Color: \textcolor{red}{Hello World}
\usepackage{color}
\usepackage{xcolor}
\usepackage{cite}
\usepackage{indentfirst}
%% URL: \url{http://stackoverflow.com/}
\usepackage{hyperref}
%% Strikeout font: \sout{Hello World} 
\usepackage[normalem]{ulem}

\begin{document}

\title{代码自查列表}
\author{Zhi Wang\\[2ex]
	zwang@nankai.edu.cn\\[2ex]
}
\date{\today{}}

\maketitle

\tableofcontents

\newpage 

\section{命名规则}

\begin{itemize}
  \item 变量命名使用名词
  \item 函数命名使用动词,或者动词+名词的格式
  \item 变量命和函数命全部用小写字母,如果包含多个单词,单词之间使用下划线“{\_}”分割 
  \item 变量名和函数名不使用拼音,例如,index.js中使用reli,替换为heat{\_}map。
  \item 变量名和函数名要有实际意义,例如, index.js中使用gender
\end{itemize}

\section{编码}

\begin{itemize}
  \item python文件编码:.py文件的头部是否标明编码信息, 例如 {\#} -*- coding: utf-8 -*-
  \item html文件编码:.html文件是否标明编码格式,例如 <meta charset="utf-8" />
  \item javascript文件编码:
\end{itemize}

\section{缩进}

\begin{itemize}
  \item 统一代码的缩紧格式(包括python、html、javascript、css的代码):每次缩进2个空格,注意不要使用Tab
\end{itemize}

\section{去除冗余}

\begin{itemize}
  \item python文件:import是否有冗余
  \item python文件:是否有冗余的函数,即定义后没有被调用的函数
  \item html文件:引入的js,css文件是否有冗余
  \item 冗余函数:例如函数内部只有一行,且该行调用了其它函数
\end{itemize}

\section{语法的检查}

\begin{itemize}
  \item html文件:标签是否匹配,例如<div>与</div>的个数是否匹配
  \item html文件:网页布局是否正确,例如分几行,分几列,每部分的宽度、高度等等
  \item html文件:检查js库、css库,是否有漏掉的
  \item html文件:检查js库的路径, 注意"."和".."的区别
  \begin{itemize}
    \item 2016年11月2日
    \item html文件在web/template文件夹下,js、css在web/static文件夹下,在html文件中引入js和css需要用"..",先退回到web目录,然后再进入static目录
  \end{itemize}
  \item html文件:检查js库的引入顺序,因为js库的引入顺序错误会导致变量为定义的错误
  \begin{itemize}
    \item 2016年11月2日
    \item js代码中使用了d3对象和queue对象,但是在html文件中,d3.js和queue.js的库文件引入顺序在我们的代码之后,导致我们的代码在执行时,系统中还没有创建d3和queue对象,浏览器的console里看到对象未定义的错误
  \end{itemize}
  \item 函数式编程:对于数据集的操作,尽量不要使用for等循环结构,使用map()、filter()、reduce()等函数式编程方法
\end{itemize}


\section{文档}

\begin{itemize}
  \item 函数:解决的问题、流程图
  \item 网页:网页的布局,例如,网页分为几个容器,每个容器的位置、宽度、高度等
  \item 使用手册:告诉用户如何使用
\end{itemize}

\section{团队合作}

\begin{itemize}
  \item 每天下午5点到6点,交换检查别人的代码
  \item 下班之前将自己的代码、文档提交到Repo
\end{itemize}

\end{document}




